% This is a sample document using the University of Minnesota, Morris, Computer Science
% Senior Seminar modification of the ACM sig-alternate style to generate a simple annotated
% bibliography. The idea is that this document is fairly short, consisting of a brief description
% of your sources and how you intend to use them (or not). Most of the ``content'' of the
% generated document comes from the bibliography file, including the notes field which will
% provide the annotations.

% See https://github.com/UMM-CSci/Senior_seminar_templates for more info and to make
% suggestions and corrections.

\documentclass{sig-alternate}

\begin{document}

% --- Author Metadata here ---
%%% REMEMBER TO CHANGE THE SEMESTER AND YEAR
\conferenceinfo{UMM CSci Senior Seminar Conference, December 2013}{Morris, MN}

\title{Usability of Error Messages for Introductory Students}

\numberofauthors{1}

\author{
% The command \alignauthor (no curly braces needed) should
% precede each author name, affiliation/snail-mail address and
% e-mail address. Additionally, tag each line of
% affiliation/address with \affaddr, and tag the
% e-mail address with \email.
\alignauthor
Paul A. Schliep\\
	\affaddr{Division of Science and Mathematics}\\
	\affaddr{University of Minnesota, Morris}\\
	\affaddr{Morris, Minnesota, USA 56267}\\
	\email{schli202@morris.umn.edu}
}

\maketitle

\begin{abstract}
Error messages are an important tool for programmers to help find and fix mistakes or issues in their code.
When an error message is unhelpful, it can be difficult to find the issue and may impose learning difficulties.
Error messages are especially critical for introductory programmers in understanding problems with their code.
Not all error messages are beneficial for helping novice programmers, however.
This paper discusses the general usability of error messages for introductory programmers, analyses of error messages in compilers and DrRacket, and two methodologies meant to improve error handling.
\end{abstract}

\keywords{Novice programmers, usability, error messages, usability studies, compiler errors, syntax errors}

\section{Key Points}

The question being addressed in this research is to figure out how well error messages in programming languages help introductory students find mistakes in their programs. This question is being addressed through various methods of evaluating the usability of the error messages and the methodologies in which they can be or potentially have been improved. The methods of evaluating the error messages include usability studies, trials, and developed rubric to assess them, and some of the methodologies that suggested improvements to the messages include a set of suggested principles for compiler errors, enhancement of syntax errors, and proposed recommendations for exception messages.

The sources for this paper each propose or analyze something new to help address this problem while still being related. On Compiler Error Messages: What They Say and What They Mean discuss the current state of compiler error messages, how that affects novice programmers, and what they suggest to resolve those issues. Enhancing Syntax Error Messages Appears Ineffectual talks about the implications of syntax error messages are on how introductory CS students can fix errors in their programs and they present an enhanced library of compiler messages specifically for syntax errors and analyze its effectiveness. Mind Your Language: On Novices’ Interactions with Error Messages and Measuring the Effectiveness of Error Messages Designed for Novice Programmers both evaluate error messages in the Racket programming language and discuss how well they help introductory students fix mistakes in their programs, and how to apply their developed rubric to improve these error messages. Each paper discusses an analysis of different types of error messages and in what ways they may be resolved, and by studying these analyses and their respective methodologies, we can be able to compare and contrast their methods and results. 

Error message usability has been tested for many years, but it has gained more attraction in recent years with the rising studies in HCI. Through these studies, developers and professors have been able to implement error message libraries and test them, but not always with conclusive results as seen in my source, Enhancing Syntax Error Messages Appears Ineffectual where the number of students’ syntax errors did not fall. 

In order to explain some of the analyses of the error messages, the reader will need to be informed of concepts such as usability studies and compiler errors.

% The following two commands are all you need to
% produce the bibliography for the citations in your paper.
\bibliographystyle{abbrv}
% annotated_bibliography.bib is the name of the BibTex file containing 
% all the bibliography entries for this example. Note that you *don't* include the .bib ending
% in the \bibliography command.
\bibliography{annotated_bibliography}  
~\cite{Denny:2014:ESE:2591708.2591748}
~\cite{Hartmann:2010:OPS:1753326.1753478}
~\cite{Isa:1983:MOE:800045.801583}
~\cite{Kummerfeld:2003:NBF:858403.858416}
~\cite{Marceau:2011:MEE:1953163.1953308}
~\cite{Marceau:2011:MYL:2048237.2048241}
~\cite{Murphy:2008:BTD:1352135.1352193}
~\cite{Traver:2010}
~\cite{Denny:2011:USB:1999747.1999807}

% You must have a ".bib" file and remember to run:
%     pdflatex bibtex pdflatex pdflatex
% in order to see all the citation references correctly.

% That's all folks!
\end{document}



