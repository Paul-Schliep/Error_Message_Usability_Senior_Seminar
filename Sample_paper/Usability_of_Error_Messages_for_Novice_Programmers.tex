% This is a sample document using the University of Minnesota, Morris, Computer Science
% Senior Seminar modification of the ACM sig-alternate style. Much of this content is taken
% directly from the ACM sample document illustrating the use of the sig-alternate class. Certain
% parts that we never use have been removed to simplify the example, and a few additional
% components have been added.

% See https://github.com/UMM-CSci/Senior_seminar_templates for more info and to make
% suggestions and corrections.

\documentclass{sig-alternate}
\usepackage{color}
\usepackage[colorinlistoftodos]{todonotes}

%%%%% Uncomment the following line and comment out the previous one
%%%%% to remove all comments
%%%%% NOTE: comments still occupy a line even if invisible;
%%%%% Don't write them as a separate paragraph
%\newcommand{\mycomment}[1]{}

\begin{document}

% --- Author Metadata here ---
%%% REMEMBER TO CHANGE THE SEMESTER AND YEAR
\conferenceinfo{UMM CSci Senior Seminar Conference, December 2013}{Morris, MN}

\title{Usability of Error Messages for Introductory Computer Science Students}

\numberofauthors{1}

\author{
% The command \alignauthor (no curly braces needed) should
% precede each author name, affiliation/snail-mail address and
% e-mail address. Additionally, tag each line of
% affiliation/address with \affaddr, and tag the
% e-mail address with \email.
\alignauthor
Paul A. Schliep\\
	\affaddr{Division of Science and Mathematics}\\
	\affaddr{University of Minnesota, Morris}\\
	\affaddr{Morris, Minnesota, USA 56267}\\
	\email{schli202@morris.umn.edu}
}

\maketitle
\begin{abstract}
This paper discusses how well error messages can help novice programmers find mistakes in their programs and what aspects make an error message more user-friendly for introductory computer science students. I will also be discussing several methodologies and programs developed to help improve the experience a novice programmer has when attempting to understand causes for errors.

% The current paper format *only* allows inline comments using the todo
% macro. That's kind of a bummer, and it would be neat if someone figured
% out how to change the acmconf style to allow this. I suspect it isn't *hard*
% but there are quite a few details that have to be sorted out in synchrony.
\todo[inline]{Needs more work}
\end{abstract}

\keywords{Novice programmers, usability, error messages, user-studies}


\section{Introduction}\label{intro}
Introduce why error messages are important, how they help introductory CS students, and outline what this paper will cover.

\todo[inline, color=red]{Write section}


\section{Background}\label{background}
This is where I will describe the concepts such as HCI, usability studies, and error message types needed to explain the research done on error message usability. I will use multiple sources from primary and background to describe these concepts. 

\todo[inline, color=red]{Write section}


\section{Analyses}\label{analyses}

\todo[inline, color=red]{Write section}


\subsection{Analysis of exception messages}
This will be a combination of two main sources used for this part: Mind Your Language: On Novices’ Interactions with Error Messages and Measuring the Effectiveness of Error Messages Designed for Novice Programmers

\subsection{Analysis of compiler messages}
This will be a combination of two main sources used for this part: On Compiler Error Messages: What They Say and some of Enhancing Syntax Error Messages Appears Ineffectual


\section{Results of case studies}\label{results}
This will discuss the studies done on error message usability for introductory students and their results and what those results mean

\todo[inline, color=red]{Write section}


\section{Methodologies}
Introduce methodologies meant to be used to improve error message libraries

\todo[inline, color=red]{Write section}

\subsection{Principles of compiler error design}
This will use my source, On Compiler Error Messages: What They Say and What They Mean

\subsection{Recommendations for error messages}
This will use my source, Mind Your Language: On Novices’ Interactions 

\subsection{Syntax error message enhancement and results}
This will use my source, Enhancing Syntax Error Messages Appears Ineffectual


\section{Conclusion}
Discussion of the direction usability of error messages will be taken and how the methodologies will be applied for future work.

\todo[inline, color=red]{Write section}


\section{Acknowledgments}

\todo[inline, color=red]{Write section}


% The following two commands are all you need in the
% initial runs of your .tex file to
% produce the bibliography for the citations in your paper.
\bibliographystyle{acm}
% sample_paper.bib is the name of the BibTex file containing the
% bibliography entries. Note that you *don't* include the .bib ending here.
\bibliography{Usability_of_Error_Messages_for_Novice_Programmers}  
% You must have a proper ".bib" file
%  and remember to run:
% latex bibtex latex latex
% to resolve all references

\end{document}
