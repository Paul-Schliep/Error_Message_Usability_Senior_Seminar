\documentclass{beamer}

\mode<presentation>
{
  \usetheme{Warsaw}
  \setbeamercovered{transparent}
}

\usepackage[english]{babel}
\usepackage[latin1]{inputenc}
\usepackage{times}
\usepackage[T1]{fontenc} 
% Or whatever. Note that the encoding and the font should match. If T1
% does not look nice, try deleting the line with the fontenc.
\usepackage{amsmath}

\newcommand{\linespace}{\vskip 0.25cm}

\definecolor{MyForestGreen}{rgb}{0,0.7,0} 
\newcommand{\tableemph}[1]{{#1}}
\newcommand{\tablewin}[1]{\tableemph{#1}}
\newcommand{\tablemid}[1]{\tableemph{#1}}
\newcommand{\tablelose}[1]{\tableemph{#1}}

\definecolor{MyLightGray}{rgb}{0.6,0.6,0.6}
\newcommand{\tabletie}[1]{\color{MyLightGray} {#1}}

% The text in square brackets is the short version of your title and will be used in the
% header/footer depending on your theme.
\title[Usability of error messages for introductory students]{Usability of error messages for \\ introductory students}

% Sub-titles are optional - uncomment and edit the next line if you want one.
% \subtitle{Why does sub-tree crossover work?} 

% The text in square brackets is the short version of your name(s) and will be used in the
% header/footer depending on your theme.
\author[Schliep]{Paul Andrew Schliep}

% The text in square brackets is the short version of your institution and will be used in the
% header/footer depending on your theme.
\institute[U of Minn, Morris]
{
  Division of Science and Mathematics \\
  University of Minnesota, Morris \\
  Morris, Minnesota, USA
}

% The text in square brackets is the short version of the date if you need that.
\date[April '15] % (optional)
{25 April 2015 \\ University of Minnesota, Morris}

% Delete this, if you do not want the table of contents to pop up at
% the beginning of each subsection:
\AtBeginSection[]
{
  \begin{frame}<beamer>
    \frametitle{Outline}
    \tableofcontents[currentsection, hideothersubsections]
  \end{frame}
}

\begin{document}

\begin{frame}
  \titlepage
\end{frame}

% For a 20-25 minute senior seminar talk you probably want something like:
% - Two or three major sections (other than the summary).
% - At *most* three subsections per section.
% - Talk about 30s to 2min per frame. So there should probably be between
%   15 and 30 frames, all told.

\section*{Introduction}

\subsection*{Introduction to error messages}

\begin{frame}
  \frametitle{Importance of error messages}
  \begin{itemize}
  	\item Error messages are important tool for beginner programmers
  	\begin{itemize}
  		\item They are the main interaction between the system and the user
  	\end{itemize}
  	\item Unhelpful error messages impose learning difficulties, especially for new programmers
  	\item Error messages with poor usability can lead the user down the wrong path
  \end{itemize}
\end{frame}

\begin{frame}
  \frametitle{Goals of an error message}
  \begin{itemize}
  	\item Starting...
  \end{itemize}
\end{frame}

\begin{frame}
  \frametitle{Some title}
  \begin{itemize}
  	\item Starting...
  \end{itemize}
\end{frame}

\subsection*{Outline}

\begin{frame}
  \frametitle{Outline}
  \tableofcontents[hideallsubsections]
\end{frame}

\section[Background]{Background}

\subsection{Type errors}

\begin{frame}
  \frametitle{Statically typed}
	\begin{itemize}
		\item Something goes here...
	\end{itemize}
\end{frame}

\begin{frame}
  \frametitle{Dynamically typed}
	\begin{itemize}
		\item Something goes here...
	\end{itemize}
\end{frame}

\subsection{Error Types}

\begin{frame}
	\frametitle{Compiler errors}
		\begin{itemize}
			\item Something goes here...
		\end{itemize}
\end{frame}

\begin{frame}
	\frametitle{Runtime errors}
		\begin{itemize}
			\item Something goes here...
		\end{itemize}

\end{frame}

\section[Analyses]{Analyses of error messages}

\subsection[DrRacket Analysis]{Analysis of DrRacket IDE}

\begin{frame}
	\frametitle{Racket programming language}
		\begin{itemize}
			\item Something goes here...
		\end{itemize}

\end{frame}

\begin{frame}
	\frametitle{IDEs and DrRacket}
		\begin{itemize}
			\item Something goes here...
		\end{itemize}

\end{frame}

\begin{frame}
	\frametitle{Study and methods}
		\begin{itemize}
			\item Something goes here...
		\end{itemize}

\end{frame}

\begin{frame}
	\frametitle{Results}
		\begin{itemize}
			\item Something goes here...
		\end{itemize}

\end{frame}

\subsection[Compiler Analysis]{Analysis of compiler errors}

\begin{frame}
	\frametitle{C++ programming language}
		\begin{itemize}
			\item Something goes here...
		\end{itemize}

\end{frame}

\begin{frame}
	\frametitle{Study and methods}
		\begin{itemize}
			\item Something goes here...
		\end{itemize}

\end{frame}

\begin{frame}
	\frametitle{Results}
		\begin{itemize}
			\item Something goes here...
		\end{itemize}

\end{frame}

\section[Improving error messages]{Improving error messages}

\subsection[DrRacket recommendations]{Recommendations for improving IDE error messages}

\begin{frame}
	\frametitle{Introduction to recommendations}
		\begin{itemize}
			\item Something goes here...
		\end{itemize}

\end{frame}

\begin{frame}
	\frametitle{first recommendations}
		\begin{itemize}
			\item Something goes here...
		\end{itemize}

\end{frame}

\begin{frame}
	\frametitle{recommendations continued}
		\begin{itemize}
			\item Something goes here...
		\end{itemize}

\end{frame}

\begin{frame}
	\frametitle{conclusions and future work for program}
		\begin{itemize}
			\item Something goes here...
		\end{itemize}

\end{frame}

\subsection[Syntax error enhancement]{Analysis of syntax error enhancement}

\begin{frame}
	\frametitle{Java and syntax errors}
		\begin{itemize}
			\item Something goes here...
		\end{itemize}

\end{frame}

\begin{frame}
	\frametitle{How they developed the program}
		\begin{itemize}
			\item Something goes here...
		\end{itemize}

\end{frame}

\begin{frame}
	\frametitle{How they tested the program}
		\begin{itemize}
			\item Something goes here...
		\end{itemize}

\end{frame}

\begin{frame}
	\frametitle{Results of syntax enhancement}
		\begin{itemize}
			\item Something goes here...
		\end{itemize}

\end{frame}

\begin{frame}
	\frametitle{Conclusions and future work of program}
		\begin{itemize}
			\item Something goes here...
		\end{itemize}

\end{frame}


\section[Conclusions]{Conclusions}

\begin{frame}
	\frametitle{Results}
		\begin{itemize}
			\item Something goes here...
		\end{itemize}

\end{frame}

\begin{frame}
	\frametitle{Future work}
		\begin{itemize}
			\item Something goes here...
		\end{itemize}

\end{frame}

\begin{frame}
	\frametitle{Acknowledgments}
	I would like to thank the following people:
		\begin{itemize}
			\item My advisor, Elena Machkasova, for helping with my senior seminar and useful feedback
			\item Friends and family for attending
			\item Myself, for writing the paper and doing this presentation
		\end{itemize}

\end{frame}

\begin{frame}
	\frametitle{Thanks!}
	
	Thank you for your time and attention!
		
	\linespace
	\linespace
	
	Contact:  
	\begin{itemize}
		\item \texttt{schli202@morris.umn.edu}
		\item \url{github.com/Paul-Schliep}
	\end{itemize}
	
	\linespace
	\linespace
	
	\begin{center}
	{\huge Questions?}
	\end{center}
\end{frame}

\section*{References}

\begin{frame} 
	\frametitle{References} 
	
	\begin{thebibliography}{lskdjf}
	
	\bibitem{McPhee:2009:gecco}
N.~F. McPhee, E.~Crane, S.~Lahr, and R.~Poli.
\newblock Developmental Plasticity in Linear Genetic Programming.
\newblock In G\"unther Raidl, \emph{et al}, editors, {\em GECCO '09}, pages 1019--1026, Montr\'eal, Qu\'ebec, Canada, 2009.
	
	\bibitem{citeulike:3452411}
	R.~Poli and N.~McPhee.
\newblock A linear estimation-of-distribution {GP} system.
\newblock In M.~O'Neill, \emph{et al}, editors, {\em EuroGP 2008}, volume
  4971 of {\em LNCS}, pages 206--217, Naples,
  26-28 Mar. 2008. Springer.
  
  	\end{thebibliography}
	
	\linespace
	\begin{center}
	See the GECCO '09 paper for additional references.
	\end{center}
\end{frame} 

\end{document}


