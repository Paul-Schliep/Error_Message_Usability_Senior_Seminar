\documentclass{beamer}

\mode<presentation>
{
  \usetheme{Warsaw}
  \setbeamercovered{transparent}
}

\usepackage[english]{babel}
\usepackage[latin1]{inputenc}
\usepackage{times}
\usepackage[T1]{fontenc} 
% Or whatever. Note that the encoding and the font should match. If T1
% does not look nice, try deleting the line with the fontenc.
\usepackage{amsmath}

\newcommand{\linespace}{\vskip 0.25cm}

\definecolor{MyForestGreen}{rgb}{0,0.7,0} 
\newcommand{\tableemph}[1]{{#1}}
\newcommand{\tablewin}[1]{\tableemph{#1}}
\newcommand{\tablemid}[1]{\tableemph{#1}}
\newcommand{\tablelose}[1]{\tableemph{#1}}

\definecolor{MyLightGray}{rgb}{0.6,0.6,0.6}
\newcommand{\tabletie}[1]{\color{MyLightGray} {#1}}

% The text in square brackets is the short version of your title and will be used in the
% header/footer depending on your theme.
\title[Usability of error messages for introductory students]{Usability of error messages for \\ introductory students}

% Sub-titles are optional - uncomment and edit the next line if you want one.
% \subtitle{Why does sub-tree crossover work?} 

% The text in square brackets is the short version of your name(s) and will be used in the
% header/footer depending on your theme.
\author[Schliep]{Paul Andrew Schliep}

% The text in square brackets is the short version of your institution and will be used in the
% header/footer depending on your theme.
\institute[U of Minn, Morris]
{
  Division of Science and Mathematics \\
  University of Minnesota, Morris \\
  Morris, Minnesota, USA
}

% The text in square brackets is the short version of the date if you need that.
\date[April '15] % (optional)
{25 April 2015 \\ University of Minnesota, Morris}

% Delete this, if you do not want the table of contents to pop up at
% the beginning of each subsection:
\AtBeginSection[]
{
  \begin{frame}<beamer>
    \frametitle{Outline}
    \tableofcontents[currentsection, hideothersubsections]
  \end{frame}
}

\begin{document}

\begin{frame}
  \titlepage
\end{frame}

% For a 20-25 minute senior seminar talk you probably want something like:
% - Two or three major sections (other than the summary).
% - At *most* three subsections per section.
% - Talk about 30s to 2min per frame. So there should probably be between
%   15 and 30 frames, all told.

\section*{Introduction}

\subsection*{Overview of error messages}

\begin{frame}[fragile]
  \frametitle{Introduction to error messages}
  \begin{itemize}
  	\item In programming, an error is when the computer cannot understand an expression in the code
  	\begin{itemize}
  		\item these errors will return an error message
  	\end{itemize}
  	\item Here's an example of an error message:
  	  \begin{verbatim}
  print("Hello World";
  ->java.3: error: unclosed string literal
  	\end{verbatim}
  \end{itemize}
\end{frame}

\begin{frame}
  \frametitle{Importance of error messages}
  \begin{itemize}
  	\item Error messages are important tool for beginner programmers
  	\begin{itemize}
  		\item one of the primary interactions between the system and the user
  	\end{itemize}
  	\item Unhelpful error messages impose learning difficulties, especially for new programmers
  	\item Error messages with poor usability can lead the user down the wrong path
  \end{itemize}
\end{frame}

\begin{frame}[fragile]
  \frametitle{Goals of an error message}
  \begin{itemize}
  	\item An error message should:
  	\begin{itemize}
  		\item not add confusion
  		\item be easy to understand
  		\item help a student locate the issue
  	\end{itemize}
  	\item Example:
  	\begin{verbatim}
  	Developing...
  	\end{verbatim}
  \end{itemize}
\end{frame}

\begin{frame}
 \frametitle{Analyzing error messages}
 \begin{itemize}
 	\item Human-computer interaction: study on interfaces between user and programs
 	\item Much of the research presented from an HCI perspective
 	\item We will discuss error messages in terms of usability
 \end{itemize}
\end{frame}

\subsection*{Outline}

\begin{frame}
  \frametitle{Outline}
  \tableofcontents[hideallsubsections]
\end{frame}

\section[Background]{Background}

\subsection{Compiler and runtime errors}

\begin{frame}[fragile]
	\frametitle{Compiler errors}
		\begin{itemize}
			\item When a compiler fails to compile a program, a user will receive a compiler error message
			\item For newer programmers, these typically occur from syntax errors
			\item Example (in Java):
			\begin{verbatim}
		int seven = (2 + 5;
		error: ')' expected
			\end{verbatim}
		\end{itemize}
\end{frame}

\begin{frame}[fragile]
	\frametitle{Runtime errors}
		\begin{itemize}
			\item A runtime error occurs after a program has compiled
			\item Usually indication of logical errors in the code
			\item Cannot be predicted, dependent on the values
			\item Example:
			\begin{verbatim}
			String string = "Hello World";
System.out.print(string.substring(6,12));

java.lang.StringIndexOutOfBoundsException:
String index out of range: 12
			\end{verbatim}
		\end{itemize}
\end{frame}

\subsection{Dynamic and statically typed}

\begin{frame}[fragile]
  \frametitle{Statically typed}
	\begin{itemize}
		\item All variables and/or objects assigned types
		\item Type checking done at compile time
		\begin{itemize}
			\item this means different error messages
		\end{itemize}
		\item Languages like Java or C++ are statically typed
		\item The following example would give an error at compile time in statically typed:
		\begin{verbatim}
		personName = "Frank"
		personName = 7
		\end{verbatim}
	\end{itemize}
\end{frame}

\begin{frame}[fragile]
  \frametitle{Dynamically typed}
	\begin{itemize}
		\item Values are not assigned to types
		\item Type checking done at runtime
		\item Languages in Lisp family
		\item The following example would give an error at runtime in dynamically typed:
		\begin{verbatim}
		personName = "Frank"
		personName = 7
		\end{verbatim}
	\end{itemize}
\end{frame}

\section[Analyses]{Analyses of error messages}

\subsection[DrRacket Analysis]{Analysis of DrRacket IDE}

\begin{frame}[fragile]
	\frametitle{Racket programming language}
		\begin{itemize}
			\item Programming language useful for teaching in introductory courses
			\item Member of Lisp languages
			\item Functional and dynamic language
			\item Syntax example:
			\begin{verbatim}
			(+ 1 2)
			-> 3
			\end{verbatim}
		\end{itemize}

\end{frame}

\begin{frame}
	\frametitle{Integrated development environments}
		\begin{itemize}
			\item An integrated development environment (IDE) is a program for writing and running code
			\item Some IDEs come packaged with debugging tools and custom error messages
		\end{itemize}

\end{frame}

\begin{frame}[fragile]
	\frametitle{DrRacket}
		\begin{itemize}
			\item An IDE for developing programs in Racket
			\item Geared toward introductory programmers
			\item DrRacket offers (mostly) user-friendly error messages and libraries to program in various levels
			%\includegraphics[keepaspectratio, width= 1.0 \textwidth]{drracketGUI.png}
		\end{itemize}
		
\end{frame}

\begin{frame}
	\frametitle{DrRacket interface}
			\includegraphics[keepaspectratio, width= 1.0 \textwidth]{drracketGUI.png}

\end{frame}

\begin{frame}
	\frametitle{Study of DrRacket error messages}
		\begin{itemize}
			\item Marceau et al. interested in finding which errors students struggled with
			\item Configured DrRacket to save a copy of each program a student tried to execute and the error messages received
			\item Programs taken from a once-per-week lab session
		\end{itemize} 

\end{frame}

\begin{frame}
	\frametitle{Table of results}
	\includegraphics[keepaspectratio, width=1.0 \textwidth]{MEE-data.pdf}

\end{frame}


\begin{frame}[fragile]
	\frametitle{Results}
		\begin{itemize}
			\item Students struggle with certain errors relative to skill level
			\item Some errors were not indicator of underlying issue
			\begin{itemize}
			\item student struggled with these errors
			\item suggests issues in error message effectiveness
			\end{itemize}
		\end{itemize}

\end{frame}

\begin{frame}[fragile]
 \frametitle{Student code example}
 			\begin{verbatim}
(define (label-near? name bias word1 word2)
  (cond
      (and (cond [(string=? name word1) 
                    "Name Located"]
                 [(string=? bias word2)
                    "Bias Located"])
           (cond [(string=? name word2)
                    "Name Located"]
                 [(string=? bias word2) 
                    "Bias Located"])
     "Mark")
))

-> and: found a use of 'and' that does not follow 
an open parenthesis
			\end{verbatim}
\end{frame}

\subsection[Compiler Analysis]{Analysis of compiler errors}

\begin{frame}
	\frametitle{C++ programming language}
		\begin{itemize}
			\item Something goes here...
		\end{itemize}

\end{frame}

\begin{frame}
	\frametitle{Study and methods}
		\begin{itemize}
			\item Something goes here...
		\end{itemize}

\end{frame}

\begin{frame}
	\frametitle{Results}
		\begin{itemize}
			\item Something goes here...
		\end{itemize}

\end{frame}

\section[Methodologies]{Methodologies for improving error messages}

\subsection[DrRacket recommendations]{Recommendations for improving IDE error messages}

\begin{frame}
	\frametitle{Introduction to recommendations}
		\begin{itemize}
			\item Something goes here...
		\end{itemize}

\end{frame}

\begin{frame}
	\frametitle{Recommendations}
		\begin{itemize}
			\item Something goes here...
		\end{itemize}

\end{frame}

\begin{frame}
	\frametitle{recommendations continued}
		\begin{itemize}
			\item Something goes here...
		\end{itemize}

\end{frame}

\begin{frame}
	\frametitle{conclusions and future work for program}
		\begin{itemize}
			\item Something goes here...
		\end{itemize}

\end{frame}

\subsection[Syntax error enhancement]{Analysis of syntax error enhancement}

\begin{frame}
	\frametitle{Java and syntax errors}
		\begin{itemize}
			\item Something goes here...
		\end{itemize}

\end{frame}

\begin{frame}
	\frametitle{How they developed the program}
		\begin{itemize}
			\item Something goes here...
		\end{itemize}

\end{frame}

\begin{frame}
	\frametitle{How they tested the program}
		\begin{itemize}
			\item Something goes here...
		\end{itemize}

\end{frame}

\begin{frame}
	\frametitle{Results of syntax enhancement}
		\begin{itemize}
			\item Something goes here...
		\end{itemize}

\end{frame}

\begin{frame}
	\frametitle{Conclusions and future work of program}
		\begin{itemize}
			\item Something goes here...
		\end{itemize}

\end{frame}


\section[Conclusions]{Conclusions}

\begin{frame}
	\frametitle{Results}
		\begin{itemize}
			\item Something goes here...
		\end{itemize}

\end{frame}

\begin{frame}
	\frametitle{Future work}
		\begin{itemize}
			\item Something goes here...
		\end{itemize}

\end{frame}

\begin{frame}
	\frametitle{Acknowledgments}
	I would like to thank the following people:
		\begin{itemize}
			\item My advisor, Elena Machkasova, for helping with my senior seminar and useful feedback on my paper and presentation
			\item Stephen Adams and Jim Hall for providing useful feedback on my paper
			\item Friends and family for attending
			\item Paul Schliep as none of this would have been possible without him
		\end{itemize}

\end{frame}

\begin{frame}
	\frametitle{Thanks!}
	
	Thank you for your time and attention!
		
	\linespace
	\linespace
	
	Contact:  
	\begin{itemize}
		\item \texttt{schli202@morris.umn.edu}
		\item \url{github.com/Paul-Schliep}
	\end{itemize}
	
	\linespace
	\linespace
	
	\begin{center}
	{\huge Questions?}
	\end{center}
\end{frame}

\section*{References}

\begin{frame} 
	\frametitle{References} 
	
	\begin{thebibliography}{lskdjf}
	
	\bibitem{McPhee:2009:gecco}
N.~F. McPhee, E.~Crane, S.~Lahr, and R.~Poli.
\newblock Developmental Plasticity in Linear Genetic Programming.
\newblock In G\"unther Raidl, \emph{et al}, editors, {\em GECCO '09}, pages 1019--1026, Montr\'eal, Qu\'ebec, Canada, 2009.
	
	\bibitem{citeulike:3452411}
	R.~Poli and N.~McPhee.
\newblock A linear estimation-of-distribution {GP} system.
\newblock In M.~O'Neill, \emph{et al}, editors, {\em EuroGP 2008}, volume
  4971 of {\em LNCS}, pages 206--217, Naples,
  26-28 Mar. 2008. Springer.
  
  	\end{thebibliography}
	
	\linespace
	\begin{center}
	See the GECCO '09 paper for additional references.
	\end{center}
\end{frame} 

\end{document}


